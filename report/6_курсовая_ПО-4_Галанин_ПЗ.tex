\documentclass[12pt, a4paper, simple]{eskdtext}

\usepackage{hyperref}
\usepackage{env}
\usepackage{_sty/gpi_lst}
\usepackage{_sty/gpi_toc}
\usepackage{_sty/gpi_t}
\usepackage{_sty/gpi_p}
\usepackage{_sty/gpi_u}

% Код
\ESKDletter{}{К}{П}
\def \gpiDocTypeNum {81}
\def \gpiCode {\ESKDtheLetterI\ESKDtheLetterII\ESKDtheLetterIII.\gpiStudentGroupName\gpiStudentGroupNum.\gpiStudentCard~-~0\gpiDocNum~\gpiDocTypeNum~\gpiDocVersion}

\def \gpiDocTopic {ПОЯСНИТЕЛЬНАЯ ЗАПИСКА К КУРСОВОМУ ПРОЕКТУ}

% Графа 1 (наименование изделия/документа)
\ESKDcolumnI {\ESKDfontII \gpiTopic \\ \gpiDocTopic}

% Графа 2 (обозначение документа)
\ESKDsignature {\gpiCode}

% Графа 9 (наименование или различительный индекс предприятия) задает команда
\ESKDcolumnIX {\gpiDepartment}

% Графа 11 (фамилии лиц, подписывающих документ) задают команды
\ESKDcolumnXIfI {\gpiStudentSurname}
\ESKDcolumnXIfII {\gpiTeacherSurname}
\ESKDcolumnXIfV {\gpiTeacherSurname}

\begin{document}
    \input{_tex/gpi_titlePage_pz.tex}

    \ESKDthisStyle{empty}
    лист с заданием
    \newpage

    % Содержание
    \ESKDthisStyle{formII}
    \tableofcontents
    \thispagestyle{toc}
    \pagestyle{toc}
    \hspace{0pt}\\
    ПРИЛОЖЕНИЕ А. НАБОР ТЕСТОВЫХ ЗАДАНИЙ ДЛЯ ПРОВЕРКИ\\
    ПРИЛОЖЕНИЕ Б. РЕЗУЛЬТАТЫ СОЗДАНИЯ, ЗАГРУЗКИ И ПРОВЕРКИ БД\\
    \newpage

    %
    \newpage
    \addcontentsline{toc}{section}{ВВЕДЕНИЕ}
    \section*{ВВЕДЕНИЕ}
    \newpage

    %
    \section{ОПИСАНИЕ МОДЕЛИ ОБЪЕКТА АВТОМАТИЗАЦИИ}
    \subsection{Организационная модель}

    Организационная структура - совокупность подразделений организации и их взаимосвязей,
    в рамках которой между подразделениями распределяются функциональные задачи,
    определяются полномочия и ответственность руководителей и должностных лиц.
    
    Структура предприятия устанавливается исходя из объема и содержания задач,
    решаемых предприятием, направленности и интенсивности сложившихся на предприятии
    информационных и документационных потоков и с учетом его организационных и материальных возможностей.

    Оргструктура представляется через органограмму и такие документы, как штатное расписание,
    устав организации и пр.

    Органограмма - графическое представление структуры организации.

    Организационная модель ОА <<Косметический салон>> представлена органограммой <<Косметический салон>>
    на рис.~\ref{fig:OrganizationnayModel}
    с использованием нотации Organizational chart методологии ARIS,
    а также таблицей <<Каталог организационных единиц>>
    на рис.~\ref{fig:OrganizationnieEdinici_katalog}.

    \begin{figure}[!h]
        \centering
        \includegraphics[height=6cm]
            {_docs/ОрганизационнаяМодель.png}
        \caption{Органограмма ОА <<Косметический салон>>}
        \label{fig:OrganizationnayModel}
    \end{figure}

    \begin{figure}[!h]
        \centering
        \includegraphics[width=14cm]
            {_docs/ОрганизационныеЕдиницы_каталог.jpg}
        \caption{Каталог организационых единиц}
        \label{fig:OrganizationnieEdinici_katalog}
    \end{figure}

    \newpage
    \subsection{Функциональная модель}

    Функциональная модель объекта автоматизации - описание его на языке выполняемых функций и их отношений.
    Функциональная структура - структура, элементами которой являются функции,
    реализуемые подразделениями предприятия, а отношениями являются связи,
    обеспечивающие передачу между элементами предметов труда.
    Функция – это предметно-ориентированное задание или действие,
    в результате которой выполняется одна или несколько целей, стоящих перед компанией.
    Функции предприятия распределяются по компонентам оргструктуры и представляют собой иерархическое дерево,
    строящееся от общего к частному.
    На самом верхнем уровне описываются самые сложные функции,
    которые потом детализируются через свои функциональные составляющие.

    Функциональная модель ОА <<Косметический салон>> представлена
    на рис.~\ref{fig:FynctionalnayModel}
    с использованием нотации Process landscape методологии ARIS,
    а также таблицей <<Каталог функций>>
    на рис.~\ref{fig:Fynctii_katalog}.

    \begin{figure}[!h]
        \centering
        \includegraphics[height=8cm]
            {_docs/ФункциональнаяМодель.png}
        \caption{Функциональное дерево ОА "Косметический салон"}
        \label{fig:FynctionalnayModel}
    \end{figure}

    \begin{figure}[!h]
        \centering
        \includegraphics[width=12cm]
            {_docs/Функции_каталог.jpg}
        \caption{Каталог функций}
        \label{fig:Fynctii_katalog}
    \end{figure}

    \newpage
    \subsection{Информационная модель}
    Информационная модель - модель объекта, представленная в виде информации,
    описывающей существенные для данного рассмотрения параметры и переменные величины объекта,
    связи между ними, входы и выходы объекта и позволяющая путём подачи на модель информации об изменениях
    входных величин моделировать возможные состояния объекта.

    Информационная модель ОА <<Инвентаризация>> для ИС <<Косметический салон>> включает в себя следующие документы:

    \begin{enumerate}
        \item[1.] Справочные документы (рисунок~\ref{fig:CP_katalog}).
        \item[2.] Оперативные документы (рисунок~\ref{fig:OP_katalog}).
        \item[3.] Отчётные документы (рисунок~\ref{fig:OT_katalog}).
    \end{enumerate}

    % Справочные документы представлены в <<Каталоге справочных документов>> .

    \begin{figure}[!h]
        \centering
        \includegraphics[width=14cm]
            {_docs/СП_каталог.jpg}
        \caption{Каталог справочных документов}
        \label{fig:CP_katalog}
    \end{figure}

    % Оперативные документы представлены в <<Каталоге оперативных документов>> .
    
    \begin{figure}[!h]
        \centering
        \includegraphics[width=14cm]
            {_docs/ОП_каталог.jpg}
        \caption{Каталог оперативных документов}
        \label{fig:OP_katalog}
    \end{figure}

    % Отчётные документы представлены в <<Каталоге отчётных документов>> .
    
    \begin{figure}[!h]
        \centering
        \includegraphics[width=14cm]
            {_docs/ОТ_каталог.jpg}
        \caption{Каталог отчётных документов}
        \label{fig:OT_katalog}
    \end{figure}

    \newpage

    \subsubsection{Справочный документ <<Номенклатура>>}

    Справочник <<Номенклатура>> - содержит информацию о материалах.
    Документ представлен в виде словаря данных (рисунок~\ref{fig:CP_Nomenkl_tipi})
    и макета (рисунок~\ref{fig:CP_Nomenkl_maket}).

    \begin{figure}[!h]
        \centering
        \includegraphics[width=14cm]
            {_docs/СП_Номенкл_типы.jpg}
        \caption{Словарь данных справочника <<Номеклатура>>}
        \label{fig:CP_Nomenkl_tipi}
    \end{figure}

    \begin{figure}[!h]
        \centering
        \includegraphics[]
            {_docs/СП_Номенкл_макет.jpg}
        \caption{Макет справочника <<Номеклатура>>}
        \label{fig:CP_Nomenkl_maket}
    \end{figure}

    \subsubsection{Справочный документ <<Сотрудники>>}

    Справочник <<Сотрудники>> - содержит информацию о сотрудниках.
    Документ представлен в виде словаря данных (рисунок~\ref{fig:CP_Sotr_tipi})
    и макета (рисунок~\ref{fig:CP_Sotr_maket}).

    \begin{figure}[!h]
        \centering
        \includegraphics[width=14cm]
            {_docs/СП_Сотр_типы.jpg}
        \caption{Словарь данных справочника <<Сотрудники>>}
        \label{fig:CP_Sotr_tipi}
    \end{figure}

    \begin{figure}[!h]
        \centering
        \includegraphics[]
            {_docs/СП_Сотр_макет.jpg}
        \caption{Макет справочника <<Сотрудники>>}
        \label{fig:CP_Sotr_maket}
    \end{figure}

    \subsubsection{Справочный документ <<Должности сотрудника>>}

    Справочник <<Должности сотрудника>> - содержит перечисления должностей сотрудника.
    Документ представлен в виде словаря данных (рисунок~\ref{fig:CP_DoljnCotr_tipi})
    и макета (рисунок~\ref{fig:CP_DoljnCotr_maket}).

    \begin{figure}[!h]
        \centering
        \includegraphics[width=14cm]
            {_docs/СП_ДолжнСотр_типы.jpg}
        \caption{Словарь данных справочника <<Должности сотрудника>>}
        \label{fig:CP_DoljnCotr_tipi}
    \end{figure}

    \begin{figure}[!h]
        \centering
        \includegraphics[]
            {_docs/СП_ДолжнСотр_макет.jpg}
        \caption{Макет справочника <<Должности сотрудника>>}
        \label{fig:CP_DoljnCotr_maket}
    \end{figure}

    \subsubsection{Справочный документ <<Единицы хранения>>}

    Справочник <<Единицы хранения>> - содержит перечисление единиц хранения.
    Документ представлен в виде словаря данных (рисунок~\ref{fig:CP_EdXran_tipi})
    и макета (рисунок~\ref{fig:CP_EdXran_maket}).

    \begin{figure}[!h]
        \centering
        \includegraphics[width=14cm]
            {_docs/СП_ЕдХран_типы.jpg}
        \caption{Словарь данных справочника <<Единицы хранения>>}
        \label{fig:CP_EdXran_tipi}
    \end{figure}

    \begin{figure}[!h]
        \centering
        \includegraphics[]
            {_docs/СП_ЕдХран_макет.jpg}
        \caption{Макет справочника <<Единицы хранения>>}
        \label{fig:CP_EdXran_maket}
    \end{figure}

    \newpage

    \subsubsection{Справочный документ <<Мои организации>>}

    Справочник <<Мои организации>> - содержит информацию о моих организациях.
    Документ представлен в виде словаря данных (рисунок~\ref{fig:CP_MoiOrg_tipi})
    и макета (рисунок~\ref{fig:CP_MoiOrg_maket}).

    \begin{figure}[!h]
        \centering
        \includegraphics[width=14cm]
            {_docs/СП_МоиОрг_типы.jpg}
        \caption{Словарь данных справочника <<Мои организации>>}
        \label{fig:CP_MoiOrg_tipi}
    \end{figure}

    \begin{figure}[!h]
        \centering
        \includegraphics[]
            {_docs/СП_МоиОрг_макет.jpg}
        \caption{Макет справочника <<Мои организации>>}
        \label{fig:CP_MoiOrg_maket}
    \end{figure}

    \subsubsection{Справочный документ <<Производители>>}

    Справочник <<Производители>> - содержит перечисление производителей.
    Документ представлен в виде словаря данных (рисунок~\ref{fig:CP_Proizv_tipi})
    и макета (рисунок~\ref{fig:CP_Proizv_maket}).

    \begin{figure}[!h]
        \centering
        \includegraphics[width=14cm]
            {_docs/СП_Произв_типы.jpg}
        \caption{Словарь данных справочника <<Производители>>}
        \label{fig:CP_Proizv_tipi}
    \end{figure}

    \begin{figure}[!h]
        \centering
        \includegraphics[]
            {_docs/СП_Произв_макет.jpg}
        \caption{Макет справочника <<Производители>>}
        \label{fig:CP_Proizv_maket}
    \end{figure}

    \newpage
    \subsubsection{Справочный документ <<Места хранения>>}

    Справочник <<Места хранения>> - содержит перечисление адреса с номером склада.
    Документ представлен в виде словаря данных (рисунок~\ref{fig:CP_MestaXran_tipi})
    и макета (рисунок~\ref{fig:CP_MestaXran_maket}).

    \begin{figure}[!h]
        \centering
        \includegraphics[width=14cm]
            {_docs/СП_МестаХран_типы.jpg}
        \caption{Словарь данных справочника <<Места хранения>>}
        \label{fig:CP_MestaXran_tipi}
    \end{figure}

    \begin{figure}[!h]
        \centering
        \includegraphics[]
            {_docs/СП_МестаХран_макет.jpg}
        \caption{Макет справочника <<Места хранения>>}
        \label{fig:CP_MestaXran_maket}
    \end{figure}

    \newpage
    \subsubsection{Оперативный документ <<Инвентаризационная опись>>}

    Оперативный документ <<Инвентаризационная опись>>
    - документ, в котором отображаются результаты инвентаризации.
    Документ представлен в виде словаря данных (рисунок~\ref{fig:OP_InvenOpis_tipi})
    и макета (рисунок~\ref{fig:OP_InvenOpis_maket}).

    \begin{figure}[!h]
        \centering
        \includegraphics[width=12cm]
            {_docs/ОП_ИнвенОпис_типы.jpg}
        \caption{Словарь данных документа <<Инвентаризационная опись>>}
        \label{fig:OP_InvenOpis_tipi}
    \end{figure}

    \begin{figure}[!h]
        \centering
        \includegraphics[width=14cm]
            {_docs/ОП_ИнвенОпис_макет.jpg}
        \caption{Макет документа <<Инвентаризационная опись>>}
        \label{fig:OP_InvenOpis_maket}
    \end{figure}

    \begin{figure}[!h]
        \centering
        \includegraphics[height=7cm]
            {_docs/ОП_ИнвенОпис_связи.png}
        \caption{Схема информационной связи документа <<Инвентаризационная опись>>}
        \label{fig:OP_InvenOpis_svazi}
    \end{figure}

    \newpage
    \subsubsection{Оперативный документ <<Приказ о создании инвентаризационной комиссии>>}

    Оперативный документ <<Приказ о создании инвентаризационной комиссии>>
    - документ, который формируется перед инвентаризацией товара.
    Документ представлен в виде словаря данных (рисунок~\ref{fig:OP_PrikazSozdKomInvest_tipi})
    и макета (рисунок~\ref{fig:OP_PrikazSozdKomInvest_maket}).

    \begin{figure}[!h]
        \centering
        \includegraphics[width=14cm]
            {_docs/ОП_ПриказСоздКомИнвент_типы.jpg}
        \caption{Словарь данных документа <<Приказ о создании инвентаризационной комиссии>>}
        \label{fig:OP_PrikazSozdKomInvest_tipi}
    \end{figure}

    \begin{figure}[!h]
        \centering
        \includegraphics[width=14cm]
            {_docs/ОП_ПриказСоздКомИнвент_макет.jpg}
        \caption{Макет документа <<Приказ о создании инвентаризационной комиссии>>}
        \label{fig:OP_PrikazSozdKomInvest_maket}
    \end{figure}

    \begin{figure}[!h]
        \centering
        \includegraphics[height=5cm]
            {_docs/ОП_ПриказСоздКомИнвент_связи.png}
        \caption{Схема информационной связи документа <<Приказ о создании инвентаризационной комиссии>>}
        \label{fig:OP_PrikazSozdKomInvest_svazi}
    \end{figure}

    \newpage
    \subsubsection{Отчётный документ <<Акт о нестаче товара>>}
    
    Отчётный документ <<Акт о нестаче товара>>
    - формируется после проведения инвентаризации.
    Документ представлен в виде словаря данных (рисунок~\ref{fig:OT_AktNedosTov_tipi})
    и макета (рисунок~\ref{fig:OT_AktNedosTov_maket}).

    \begin{figure}[!h]
        \centering
        \includegraphics[width=14cm]
            {_docs/ОТ_АктНедосТов_типы.jpg}
        \caption{Словарь данных документа <<Акт о нестаче товара>>}
        \label{fig:OT_AktNedosTov_tipi}
    \end{figure}

    \begin{figure}[!h]
        \centering
        \includegraphics[width=14cm]
            {_docs/ОТ_АктНедосТов_макет.jpg}
        \caption{Макет документа <<Акт о нестаче товара>>}
        \label{fig:OT_AktNedosTov_maket}
    \end{figure}

    \begin{figure}[!h]
        \centering
        \includegraphics[height=6cm]
            {_docs/ОТ_АктНедосТов_связи.png}
        \caption{Схема информационной связи документа <<Акт о нестаче товара>>}
        \label{fig:OP_AktNedosTov_svazi}
    \end{figure}

    \newpage
    \subsection{Модель бизнес-процесса объекта автоматизации}

    \textbf{Процесс} - любая деятельность, в которой используются ресурсы для преобразования входов в выходы.
    Зачастую представляет из себя совокупность взаимосвязанных и совершенных работ,
    в которых результаты одной работы являются началом другой работы,
    образуя цепочку внутрненних поставщиков и потребителей.

    \textbf{Бизнес-процесс} - устойчивая и целенаправленная совокупность взаимосвязанных видов деятельности,
    которая по определённой технологии преобразует входной сигнал в выходной, представляющий ценность для потребителя.

    \textbf{eEPC} - нотация для проектирования бизнес-процессов.
    Данная нотация ARIS представляет бизнес-процесс как цепочку событий и действий (функций).
    Каждое действие инициализируется и завершается событием.

    \begin{figure}[!h]
        \centering
        \begin{minipage}{0.15\textwidth}
            \centering
            \includegraphics[width=0.99\textwidth]
                {_docs/МБП_Event.png}
            \caption{Event}
            \label{fig:MBP_Event}
        \end{minipage}
        \begin{minipage}{0.15\textwidth}
            \centering
            \includegraphics[width=0.99\textwidth]
                {_docs/МБП_Activity.png}
            \caption{Activity}
            \label{fig:MBP_Activity}
        \end{minipage}
        \begin{minipage}{0.15\textwidth}
            \centering
            \includegraphics[width=0.99\textwidth]
                {_docs/МБП_Entity.png}
            \caption{Entity}
            \label{fig:MBP_Entity}
        \end{minipage}
        \begin{minipage}{0.15\textwidth}
            \centering
            \includegraphics[width=0.99\textwidth]
                {_docs/МБП_Document.png}
            \caption{Document}
            \label{fig:MBP_Document}
        \end{minipage}
        \begin{minipage}{0.15\textwidth}
            \centering
            \includegraphics[width=0.99\textwidth]
                {_docs/МБП_Location.png}
            \caption{Location}
            \label{fig:MBP_Location}
        \end{minipage}
        \begin{minipage}{0.15\textwidth}
            \centering
            \includegraphics[width=0.99\textwidth]
                {_docs/МБП_Product.png}
            \caption{Product}
            \label{fig:MBP_Product}
        \end{minipage}
    \end{figure}

    \textbf{Event} (см. рис.~\ref{fig:MBP_Event}) - состояние, которое является существенным для целей управления бизнесом
    и оказывает влияние или контролирует дальнейшее развитие одного или более бизнес-процессов.
    Элемент отображает события, активизирующие функции или порождаемые функциями.
    Внутри блока помещается наименование события.
    Событие именуется отглагольным существительным.

    \textbf{Activity} (см. рис.~\ref{fig:MBP_Activity}) - действие или набор действий, выполняемых над исходным объектом с целью получения заданного результата.
    Внутри блока помещается наименование функции (глагол или отглагольное существительное).
    Временная последовательность выполнения функций задается расположением функций на диаграмме процесса сверху вниз.
    Функция именуется глаголом или отглагольным существительным.

    \textbf{Location} (см. рис.~\ref{fig:MBP_Location}) - см. <<Организационная модель ОА>>.

    \textbf{Entity} (см. рис.~\ref{fig:MBP_Entity}) (cущность) - смысловая единица даталогической модели (ER-model, модель «сущность - связь»)

    \textbf{Document} (см. рис.~\ref{fig:MBP_Document}) - бумажный или электронный носитель информации.

    \textbf{Product} (см. рис.~\ref{fig:MBP_Product}) - ресурс или услуга, используемый как для входа в функцию, так и как результат ее выполнения.

    \newpage

    Модель бизнесс-процесса ОА <<Инвентаризация>> для ИС <<Косметический салон>> представлена на
    рисунке~\ref{fig:MBP}.

    \begin{figure}[!h]
        \centering
        \includegraphics[width=18cm]
            {_docs/МБП.png}
        \caption{Модель БП ОА «Инвентаризация» для ИС <<Косметический салон>>}
        \label{fig:MBP}
    \end{figure}

    % = = = = = = = = = = = = = = = =
    \newpage
    \section{РАЗРАБОТКА БАЗЫ ДАННЫХ}
    \subsection{Концептуальная модель}

    Предметная область - совокупность объектов,
    свойства которых и отношения между которыми рассматриваются в рамках некоторого исследования.
    
    Модель предметной области – некоторая система, адекватно имитирующая
    структуру и функционирование исследуемой предметной области.

    Концептуальная модель - это структура моделируемой предметной области,
    свойств её элементов и причинно-следственных связей, присущих системе и
    существенных для достижения цели моделирования.
    В рамках этапа концептуального моделирования выделяются основные смысловые единицы (сущности)
    предметной области, определяются и описываются связи между ними.
    
    Концептуальная модель ориентирована на потенциальных пользователей базы данных,
    так как представляет предметную область на их уровне понимания.
    Этот уровень называется системно-независимым или предметно-ориентированным.
    
    \subsubsection{ЛКМ для БП <<Приказ о создании комиссии инвентаризационной>>}

    Локальная концептуальная модель для бизнес процесса с документом <<Приказ о создании комиссии инвентаризационной>>
    изображена на рисунке~\ref{fig:LKM_PrikazSozdKomInvent}.

    \begin{figure}[!h]
        \centering
        \includegraphics[width=18cm]
            {_docs/ЛКМ_ПриказСоздКомИнвент.png}
        \caption{Локальная концептуальная модель для бизнес процесса с документом <<Приказ о создании комиссии инвентаризационной>>}
        \label{fig:LKM_PrikazSozdKomInvent}
    \end{figure}

    \subsubsection{ЛКМ для БП <<Инвентаризационная опись>>}

    Локальная концептуальная модель для бизнес процесса с документом <<Инвентаризационная опись>>
    изображена на рисунке~\ref{fig:LKM_InvenOpis}.

    \begin{figure}[!h]
        \centering
        \includegraphics[width=18cm]
            {_docs/ЛКМ_ИнвенОпис.png}
        \caption{Локальная концептуальная модель для бизнес процесса с документом <<Инвентаризационная опись>>}
        \label{fig:LKM_InvenOpis}
    \end{figure}

    \newpage
    \subsubsection{КМ для БП ОА}

    Концептуальная модель для бизнес-процесса объекта автоматизации <<Инвентаризация>>
    изображена на рисунке~\ref{fig:KM}.

    \begin{figure}[!h]
        \centering
        \includegraphics[width=18cm]
            {_docs/КМ.png}
        \caption{Концептуальная модель}
        \label{fig:KM}
    \end{figure}

    \newpage
    \subsection{Логическая модель}

    Логическая модель 
    изображена на рисунке~\ref{fig:KM}.

    \begin{figure}[!h]
        \centering
        \includegraphics[width=18cm]
            {_docs/ЛМ.png}
        \caption{Логическая модель}
        \label{fig:LM}
    \end{figure}

    \newpage
    \subsection{Физическая модель}

    \lstinputlisting[language=sql]
        {src/create_database.sql}

    \lstinputlisting[language=sql]
        {src/delete_database.sql}

    %
    \newpage
    \addcontentsline{toc}{section}{ЗАКЛЮЧЕНИЕ}
    \section*{ЗАКЛЮЧЕНИЕ}
    \newpage

    %
    \newpage
    \addcontentsline{toc}{section}{СПИСОК ИСПОЛЬЗОВАННЫХ ИСТОЧНИКОВ}
    \section*{СПИСОК ИСПОЛЬЗОВАННЫХ ИСТОЧНИКОВ}
    \begin{enumerate}    
        \item[1.] Форма 401 1. Приложение 5. 2. к Инструкции по...
        - [Электронный ресурс]
        Режим доступа: \url{https://neg.by/public_files/NEG__5.XLS}
        Дата~доступа:~08.03.2022.
        \item[2.] Годовая инвентаризация 2021 (Как проводится инвентаризация)
        - [Электронный ресурс]
        Режим доступа: \url{https://www.gb.by/articles/godovaya-inventarizatsiya-2020}
        Дата~доступа:~15.03.2022.
        \item[3.] 10.docx (Приказ о назначении инвентаризационной комиссии)
        - [Электронный ресурс]
        Режим доступа: \url{https://docviewer.yandex.by/view/0/?*=bRCKBHW0FptcY0rJ%2BVjcsYyxekV7InVybCI6Imh0dHBzOi8vanVyYnVoLmJ5L2RvY3Mvb3JkZXJzLzEwLmRvY3giLCJ0aXRsZSI6IjEwLmRvY3giLCJub2lmcmFtZSI6dHJ1ZSwidWlkIjoiMCIsInRzIjoxNjQ3Mzc2NzAyMjQwLCJ5dSI6IjUyNTIwNjUwMjE2NDQ3NTk1NzciLCJzZXJwUGFyYW1zIjoidG09MTY0NzM3NjY4MyZ0bGQ9YnkmbGFuZz1ydSZuYW1lPTEwLmRvY3gmdGV4dD0lRDAlQkYlRDElODAlRDAlQjglRDAlQkElRDAlQjAlRDAlQjcrJUQwJUJFKyVEMSU4MSVEMCVCRSVEMCVCNyVEMCVCNCVEMCVCMCVEMCVCRCVEMCVCOCVEMCVCOCslRDAlQkElRDAlQkUlRDAlQkMlRDAlQjglRDElODElRDElODElRDAlQjglRDAlQjgrJUQwJUI4JUQwJUJEJUQwJUIyJUQwJUI1JUQwJUJEJUQxJTgyJUQwJUIwJUQxJTgwJUQwJUI4JUQwJUI3JUQwJUIwJUQxJTg2JUQwJUI4JUQwJUJFJUQwJUJEJUQwJUJEJUQwJUJFJUQwJUI5JnVybD1odHRwcyUzQS8vanVyYnVoLmJ5L2RvY3Mvb3JkZXJzLzEwLmRvY3gmbHI9MjE1MTAmbWltZT1kb2N4JmwxMG49cnUmc2lnbj1jNzE4NzE4YWYzNWI3MDkyOGViYTcyOWI2ZTY5OTAzNyZrZXlubz0wIn0%3D&amp;lang=ru}
        Дата~доступа:~15.03.2022.
        \item[4.] Перчатки винил/нитрил чёрные, 100шт. Купить в Молодечно по цене 17.5 руб | Отзывы на Tomas.by. ID: 300876683.
        - [Электронный ресурс]
        Режим доступа: \url{https://tomas.by/p/300876683-perchatki-vinil-nitril-chernye-100sht/}
        Дата~доступа:~10.05.2022.
        \item[5.] ПОЛОТЕНЦА ОДНОРАЗОВЫЕ СПАНЛЕЙС (35Х70 СМ) 50 ШТ.. Купить в Молодечно по цене 11.5 руб | Отзывы на Tomas.by. ID: 300305363.
        - [Электронный ресурс]
        Режим доступа: \url{https://tomas.by/p/300305363-polotenca-odnorazovye-spanleys-35h70-sm-50-sht/}
        Дата~доступа:~10.05.2022.
        \item[6.] Простыни в рулоне 80х200см, 100шт. - 23.5 руб в Молодечно. Купить по выгодной цене на Tomas.by. Отзывы, ID: 300307416.
        - [Электронный ресурс]
        Режим доступа: \url{https://tomas.by/p/300307416-prostyni-v-rulone-80h200sm-100sht/}
        Дата~доступа:~10.05.2022.
        \item[7.] Накидка парикмахерская двусторонняя, цвет серебристый/чёрный - купить по цене 22 руб в Минске | Отзывы на Tomas.by (256731020)
        - [Электронный ресурс]
        Режим доступа: \url{https://tomas.by/p/256731020-nakidka-parikmaherskaya-dvustoronnyaya-cvet-serebristyy-chernyy/}
        Дата~доступа:~10.05.2022.
        \item[8.] Маникюрная подставка для рук на металлических ножках. Белая - купить по цене 38 руб в Молодечно | Отзывы на Tomas.by (303306330)
        - [Электронный ресурс]
        Режим доступа: \url{https://tomas.by/p/303306330-manikyurnaya-podstavka-dlya-ruk-na-metallicheskih-nozhkah-belaya/}
        Дата~доступа:~10.05.2022.
        \item[9.] Салфетки 5х5см, 400шт. в Молодечно: низкие цены, доставка, отзывы - купить на Tomas.by (300307438)
        - [Электронный ресурс]
        Режим доступа: \url{https://tomas.by/p/300307438-salfetki-5h5sm-400sht/}
        Дата~доступа:~10.05.2022.
        \item[10.] Бумага под воротник Eurostil 100шт 00681. Купить в Витебской области по цене 12 руб | Отзывы на Tomas.by. ID: 235339583.
        - [Электронный ресурс]
        Режим доступа: \url{https://tomas.by/p/235339583-bumaga-pod-vorotnik-eurostil-100sht-00681/}
        Дата~доступа:~10.05.2022.
        \item[11.] Пластиковый шпатель для масок - 4.3637153211236 руб в Гомеле. Купить по выгодной цене на Tomas.by. Отзывы, ID: 282800096.
        - [Электронный ресурс]
        Режим доступа: \url{https://tomas.by/p/282800096-plastikovyy-shpatel-dlya-masok/}
        Дата~доступа:~10.05.2022.
        \item[12.] Набор для окрашивания волос 3 предмета Мой Мир - Купить в Минске. Актуальная цена - 3.7 руб на Tomas.by. Отзывы, ID: 271907887.
        - [Электронный ресурс]
        Режим доступа: \url{https://tomas.by/p/271907887-nabor-dlya-okrashivaniya-volos-3-predmeta-moy-mir/}
        Дата~доступа:~10.05.2022.
    \end{enumerate}
    \newpage

    %
    \newpage
    \addcontentsline{toc}{section}{СПИСОК СОКРАЩЕНИЙ}
    \section*{СПИСОК СОКРАЩЕНИЙ}
    
    \begin{tabular}{ll} 
        ARIS    & architecture of integrated information system.\\
        ИС      & информационная система.\\
        МОЛ     & материально отвественное лицо.\\
        ОА      & объект автоматизации.\\
        ОП      & оперативный документ.\\
        ОТ      & отчётный документ.\\
        СП      & справочный документ.\\
        % СУБД    & система управления базами данных.\\
        % ТМЦ     & товарно-материальные ценности.\\
        БД      & база данных.\\
        БП      & бизнес процесс.\\
        ЛКМ     & локальная концептуальная модель.\\ 
        КМ      & концептуальная модель.\\
    \end{tabular}
    
    \newpage
\end{document}
