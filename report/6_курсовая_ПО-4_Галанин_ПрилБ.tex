\documentclass[12pt, a4paper, simple]{eskdtext}

\usepackage{hyperref}
\usepackage{env}
\usepackage{_sty/gpi_lst}
\usepackage{_sty/gpi_toc}
\usepackage{_sty/gpi_t}
\usepackage{_sty/gpi_p}

\def \gpiPrilLetter {Б}

% Код
\ESKDletter{}{К}{П}
\def \gpiDocTypeNum {12}
\def \gpiCode {\ESKDtheLetterI\ESKDtheLetterII\ESKDtheLetterIII.\gpiStudentGroupName\gpiStudentGroupNum.\gpiStudentCard~-~0\gpiDocNum~\gpiDocTypeNum~\gpiDocVersion}

\def \gpiDocTopic {РЕЗУЛЬТАТЫ СОЗДАНИЯ, ЗАГРУЗКИ И ПРОВЕРКИ БД}

% колонтитулы
\usepackage{fancybox, fancyhdr}
\fancypagestyle{plain}
{
    \renewcommand{\footrulewidth}{0pt}          % Толщина отделяющей полоски снизу
    \renewcommand{\headrulewidth}{0pt}          % Толщина отделяющей полоски сверху
    \fancyhead[C]{\hfill\gpiCode\hfill\thepage} % Сверху по центру выводить код
    \fancyfoot{}                                % Очистить нижний колонтитул
}

% Графа 1 (наименование изделия/документа)
\ESKDcolumnI {\ESKDfontII \gpiTopic \\ \gpiDocTopic}

% Графа 2 (обозначение документа)
\ESKDsignature {\gpiCode}

% Графа 9 (наименование или различительный индекс предприятия) задает команда
\ESKDcolumnIX {\gpiDepartment}

% Графа 11 (фамилии лиц, подписывающих документ) задают команды
\ESKDcolumnXIfI {\gpiStudentSurname}
\ESKDcolumnXIfII {\gpiTeacherSurname}
\ESKDcolumnXIfV {\gpiTeacherSurname}

\renewcommand {\thefigure} {Б.\arabic{figure}}

\begin{document}
    \begin{ESKDtitlePage}
    \begin{flushright}
        \textbf{ПРИЛОЖЕНИЕ \gpiPrilLetter} \enspace\enspace
    \end{flushright}
    \begin{center}
        % \gpiMinEdu \\
        \gpiEdu \\
        \gpiKaf \\
    \end{center}

    \vfill

    \begin{center}
        \textbf{\gpiTopic} \\
    \end{center}

    \vfill

    \begin{center}
        \gpiDocTopic \\
        ПО ДИСЦИПЛИНЕ \gpiDiscipline \\
    \end{center}

    \vfill

    \begin{center}
        \textbf{\gpiCode} \\
    \end{center}

    \begin{flushright}
        \begin{minipage}[t]{.45\textwidth}
            Листов \pageref{LastPage} \\
        \end{minipage}
    \end{flushright}

    \vfill

    \begin{flushright}
        \begin{minipage}[t]{.49\textwidth}
            \begin{minipage}[t]{.75\textwidth}
                \begin{flushright}
                    Руководитель

                    \hspace{0pt}

                    Выполнил

                    \hspace{0pt}

                    Консультанты:

                    по ЕСПД

                    Рецензент
                \end{flushright}
            \end{minipage}
        \end{minipage}
        \begin{minipage}[t]{.49\textwidth}
            \begin{flushright}
                \begin{minipage}[t]{.75\textwidth}
                    \gpiTeacherName~\gpiTeacherSurname

                    \hspace{0pt}

                    \gpiStudentName~\gpiStudentSurname

                    \hspace{0pt}

                    \hspace{0pt}

                    \gpiTeacherName~\gpiTeacherSurname

                    \gpiTeacherName~\gpiTeacherSurname
                \end{minipage}
            \end{flushright}
            
        \end{minipage}
    \end{flushright}

    \vfill

    \begin{center}
        \PageTitleCity~\ESKDtheYear
    \end{center}
\end{ESKDtitlePage}


    \ESKDstyle{title}
    \thispagestyle{plain}
    \pagestyle{plain}
    \hspace{0pt}

    Таблицы в базе данных:
    \begin{itemize}
        \item \textbf{СП\_ЕдХран} - см. рис.~\ref{fig:CP_EdXran}
        \item \textbf{СП\_Произв} - см. рис.~\ref{fig:CP_Proizv}
        \item \textbf{СП\_МоиОрг} - см. рис.~\ref{fig:CP_MoiOrg}
        \item \textbf{СП\_ДолжСотр} - см. рис.~\ref{fig:CP_DoljSotr}
        \item \textbf{СП\_Сотр} - см. рис.~\ref{fig:CP_Sotr}
        \item \textbf{СП\_МестаХран} - см. рис.~\ref{fig:CP_MestaXran}
        \item \textbf{СП\_Номенкл} - см. рис.~\ref{fig:CP_Nomencl}
        \item \textbf{КомИнвент} - см. рис.~\ref{fig:CP_KomInvent}
        \item \textbf{ТабличнаяЧасть\_СоставКомис} - см. рис.~\ref{fig:TablichnajaChast_SostavKomissii}
        \item \textbf{ОП\_ИнвенОпис} - см. рис.~\ref{fig:OP_InvenOpis}
        \item \textbf{ОП\_ПриказСоздКомИнвент} - см. рис.~\ref{fig:OP_PrikazSozdKomInvent}
        \item \textbf{ТабличнаяЧасть\_Номенкл} - см. рис.~\ref{fig:TablichnajaChast_Nomencl}
    \end{itemize}

    \lstinputlisting[language=sql]
        {src/create.sql}

    \begin{figure}[!h]
        \centering
        \includegraphics[width=12cm]
            {_assets/СП_ЕдХран.png}
        \caption{Таблица справочного документа <<Единицы хранения>> в БД}
        \label{fig:CP_EdXran}
    \end{figure}

    \begin{figure}[!h]
        \centering
        \includegraphics[width=12cm]
            {_assets/СП_Произв.png}
        \caption{Таблица справочного документа <<Производители>> в БД}
        \label{fig:CP_Proizv}
    \end{figure}

    \begin{figure}[!h]
        \centering
        \includegraphics[width=12cm]
            {_assets/СП_МоиОрг.png}
        \caption{Таблица справочного документа <<Мои организации>> в БД}
        \label{fig:CP_MoiOrg}
    \end{figure}

    \begin{figure}[!h]
        \centering
        \includegraphics[width=12cm]
            {_assets/СП_ДолжСотр.png}
        \caption{Таблица справочного документа <<Должности сотрудника>> в БД}
        \label{fig:CP_DoljSotr}
    \end{figure}

    \begin{figure}[!h]
        \centering
        \includegraphics[width=12cm]
            {_assets/СП_Сотр.png}
        \caption{Таблица справочного документа <<Сотрудники>> в БД}
        \label{fig:CP_Sotr}
    \end{figure}

    \begin{figure}[!h]
        \centering
        \includegraphics[width=12cm]
            {_assets/СП_МестаХран.png}
        \caption{Таблица справочного документа <<Места хранения>> в БД}
        \label{fig:CP_MestaXran}
    \end{figure}

    \begin{figure}[!h]
        \centering
        \includegraphics[width=18cm]
            {_assets/СП_Номенкл.png}
        \caption{Таблица справочного документа <<Номенклатура>> в БД}
        \label{fig:CP_Nomencl}
    \end{figure}

    \begin{figure}[!h]
        \centering
        \includegraphics[width=12cm]
            {_assets/КомИнвент.png}
        \caption{Таблица <<Коммисия инвентаризационная>> в БД}
        \label{fig:CP_KomInvent}
    \end{figure}

    \begin{figure}[!h]
        \centering
        \includegraphics[width=12cm]
            {_assets/ТабличнаяЧасть_СоставКомис.png}
        \caption{Таблица <<Табличная часть - Состав комиcсии>> в БД}
        \label{fig:TablichnajaChast_SostavKomissii}
    \end{figure}

    \begin{figure}[!h]
        \centering
        \includegraphics[width=12cm]
            {_assets/ОП_ИнвенОпис.png}
        \caption{Таблица оперативного документа <<Инвентаризационная опись>> в БД}
        \label{fig:OP_InvenOpis}
    \end{figure}

    \begin{figure}[!h]
        \centering
        \includegraphics[width=12cm]
            {_assets/ОП_ПриказСоздКомИнвент.png}
        \caption{Таблица оперативного документа <<Приказ о создании комиссиии инвентаризационной>> в БД}
        \label{fig:OP_PrikazSozdKomInvent}
    \end{figure}

    \begin{figure}[!h]
        \centering
        \includegraphics[width=18cm]
            {_assets/ТабличнаяЧасть_Номенкл.png}
        \caption{Таблица <<Табличная часть - Номенклатура>> в БД}
        \label{fig:TablichnajaChast_Nomencl}
    \end{figure}
\end{document}
