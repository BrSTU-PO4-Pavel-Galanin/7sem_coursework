\documentclass[12pt, a4paper, simple]{eskdtext}

\usepackage{hyperref}
\usepackage{env}
\usepackage{_sty/gpi_lst}
\usepackage{_sty/gpi_toc}
\usepackage{_sty/gpi_t}
\usepackage{_sty/gpi_p}
\usepackage{_sty/gpi_u}

% Код
\ESKDletter{}{К}{П}
\def \gpiDocTypeNum {81}
\def \gpiDocVer {00}
\def \gpiCode {\ESKDtheLetterI\ESKDtheLetterII\ESKDtheLetterIII.\gpiStudentGroupName\gpiStudentGroupNum.\gpiStudentCard~-~0\gpiDocNum~\gpiDocTypeNum~\gpiDocVer}

\def \gpiDocTopic {ПОЯСНИТЕЛЬНАЯ ЗАПИСКА К КУРСОВОМУ ПРОЕКТУ}

% Графа 1 (наименование изделия/документа)
\ESKDcolumnI {\ESKDfontII \gpiTopic \\ \gpiDocTopic}

% Графа 2 (обозначение документа)
\ESKDsignature {\gpiCode}

% Графа 9 (наименование или различительный индекс предприятия) задает команда
\ESKDcolumnIX {\gpiDepartment}

% Графа 11 (фамилии лиц, подписывающих документ) задают команды
\ESKDcolumnXIfI {\gpiStudentSurname}
\ESKDcolumnXIfII {\gpiTeacherSurname}
\ESKDcolumnXIfV {\gpiTeacherSurname}

\begin{document}
    \input{_tex/gpi_titlePage_pz.tex}

    % Содержание
    \tableofcontents
    \thispagestyle{toc}
    \pagestyle{toc}
    \paragraph{ПРИЛОЖЕНИЕ А. НАБОР ТЕСТОВЫХ ЗАДАНИЙ ДЛЯ ПРОВЕРКИ}
    \paragraph{ПРИЛОЖЕНИЕ Б. РЕЗУЛЬТАТЫ СОЗДАНИЯ, ЗАГРУЗКИ И ПРОВЕРКИ БД}
    \newpage

    %
    \newpage
    \addcontentsline{toc}{section}{ВВЕДЕНИЕ}
    \section*{ВВЕДЕНИЕ}
    \newpage

    %
    \section{ОПИСАНИЕ МОДЕЛИ АВТОМАТИЗАЦИИ}
    \subsection{Организационная модель}
    \subsection{Функциональная модель}

    \newpage
    \subsection{Информационная модель}
    Информационная модель - модель объекта, представленная в виде информации,
    описывающей существенные для данного рассмотрения параметры и переменные величины объекта,
    связи между ними, входы и выходы объекта и позволяющая путём подачи на модель информации об изменениях
    входных величин моделировать возможные состояния объекта.

    Информационная модель ОА <<Инвентаризация>> для ИС <<Косметический салон>> включает в себя следующие документы:
    \begin{enumerate}
        \item[1.] Справочные документы
        \item[2.] Оперативные документы
        \item[3.] Отчётные документы
    \end{enumerate}

    \subsubsection{Справочные документы}

    Справочные документы представлены в <<Каталоге справочных документов>> (рисунок~\ref{fig:docs}).
    \begin{figure}[!h]
        \centering
        \includegraphics[]
            {_docs/docs.jpg}
        \caption{Каталог справочных документов}
        \label{fig:docs}
    \end{figure}

    Справочник <<Должности сотрудника>> - содержит должности сотрудника.
    Документ представлен в виде словаря данных (рисунок~\ref{fig:cd_DoljnCotr_tipi})
    и макета (рисунок~\ref{fig:cd_DoljnCotr_maket}).

    \begin{figure}[p!]
        \centering
        \includegraphics[]
            {_docs/СД_ДолжнСотр_типы.jpg}
        \caption{Словарь данных справочника <<Должности сотрудника>>}
        \label{fig:cd_DoljnCotr_tipi}
    \end{figure}

    \begin{figure}[p!]
        \centering
        \includegraphics[]
            {_docs/СД_ДолжнСотр_макет.jpg}
        \caption{Макет справочника <<Должности сотрудника>>}
        \label{fig:cd_DoljnCotr_maket}
    \end{figure}

    Справочник <<Единицы хранения>> - содержит единицы хранения материала.
    Документ представлен в виде словаря данных (рисунок~\ref{fig:cd_EdXran_tipi})
    и макета (рисунок~\ref{fig:cd_EdXran_maket}).

    \begin{figure}[p!]
        \centering
        \includegraphics[]
            {_docs/СД_ЕдХран_типы.jpg}
        \caption{Словарь данных справочника <<Единицы хранения>>}
        \label{fig:cd_EdXran_tipi}
    \end{figure}

    \begin{figure}[p!]
        \centering
        \includegraphics[]
            {_docs/СД_ЕдХран_макет.jpg}
        \caption{Макет справочника <<Единицы хранения>>}
        \label{fig:cd_EdXran_maket}
    \end{figure}

    Справочник <<Контрагенты>> - содержит информацию о контрагентах.
    Документ представлен в виде словаря данных (рисунок~\ref{fig:cd_Kontrag_tipi})
    и макета (рисунок~\ref{fig:cd_Kontrag_maket}).

    \begin{figure}[p!]
        \centering
        \includegraphics[width=16cm]
            {_docs/СД_Контраг_типы.jpg}
        \caption{Словарь данных справочника <<Контрагенты>>}
        \label{fig:cd_Kontrag_tipi}
    \end{figure}

    \begin{figure}[p!]
        \centering
        \includegraphics[]
            {_docs/СД_Контраг_макет.jpg}
        \caption{Макет справочника <<Контрагенты>>}
        \label{fig:cd_Kontrag_maket}
    \end{figure}

    Справочник <<Номенклатура>> - содержит информацию о материалах.
    Документ представлен в виде словаря данных (рисунок~\ref{fig:cd_Nomenkl_tipi})
    и макета (рисунок~\ref{fig:cd_Nomenkl_maket}).

    \begin{figure}[p!]
        \centering
        \includegraphics[width=16cm]
            {_docs/СД_Номенкл_типы.jpg}
        \caption{Словарь данных справочника <<Номеклатура>>}
        \label{fig:cd_Nomenkl_tipi}
    \end{figure}

    \begin{figure}[p!]
        \centering
        \includegraphics[]
            {_docs/СД_Номенкл_макет.jpg}
        \caption{Макет справочника <<Номеклатура>>}
        \label{fig:cd_Nomenkl_maket}
    \end{figure}

    Справочник <<Производители>> - содержит информацию о производителях.
    Документ представлен в виде словаря данных (рисунок~\ref{fig:cd_Proizv_tipi})
    и макета (рисунок~\ref{fig:cd_Proizv_maket}).

    \begin{figure}[!h]
        \centering
        \includegraphics[width=16cm]
            {_docs/СД_Произв_типы.jpg}
        \caption{Словарь данных справочника <<Производители>>}
        \label{fig:cd_Proizv_tipi}
    \end{figure}

    \begin{figure}[!h]
        \centering
        \includegraphics[]
            {_docs/СД_Произв_макет.jpg}
        \caption{Макет справочника <<Производители>>}
        \label{fig:cd_Proizv_maket}
    \end{figure}

    Справочник <<Сотрудники>> - содержит информацию о сотрудниках.
    Документ представлен в виде словаря данных (рисунок~\ref{fig:cd_Sotr_tipi})
    и макета (рисунок~\ref{fig:cd_Sotr_maket}).

    \begin{figure}[!h]
        \centering
        \includegraphics[]
            {_docs/СД_Сотр_типы.jpg}
        \caption{Словарь данных справочника <<Сотрудники>>}
        \label{fig:cd_Sotr_tipi}
    \end{figure}

    \begin{figure}[!h]
        \centering
        \includegraphics[]
            {_docs/СД_Сотр_макет.jpg}
        \caption{Макет справочника <<Сотрудники>>}
        \label{fig:cd_Sotr_maket}
    \end{figure}

    \subsubsection{Оперативные документы}
    \subsubsection{Отчётные документы}

    \subsection{Модель бизнес-процесса объекта автоматизации}
    \newpage

    %
    \section{РАЗРАБОТКА БАЗЫ ДАННЫХ}
    \subsection{Концептуальная модель}
    \subsection{Логическая модель}
    \subsection{Физическая модель}
    \newpage

    %
    \newpage
    \addcontentsline{toc}{section}{ЗАКЛЮЧЕНИЕ}
    \section*{ЗАКЛЮЧЕНИЕ}
    \newpage

    %
    \newpage
    \addcontentsline{toc}{section}{СПИСОК ИСПОЛЬЗОВАННЫХ ИСТОЧНИКОВ}
    \section*{СПИСОК ИСПОЛЬЗОВАННЫХ ИСТОЧНИКОВ}
    \begin{enumerate}
        \item[1.] ОСНОВЫ VAGRANT > Установка Vagrant на Windows 10 - [Электронный ресурс]
        Режим доступа: \url{https://www.youtube.com/watch?v=fESCSA-wQEQ}
        Дата~доступа:~13.02.2022.
        \item[2.] Установка Apache, PHP, MySQL (LAMP) на VDS сервер (в Ubuntu) - [Электронный ресурс]
        Режим доступа: \url{https://www.youtube.com/watch?v=FxwPQkP3OGY}
        Дата~доступа:~15.02.2022.
    \end{enumerate}
    \newpage

    %
    \newpage
    \addcontentsline{toc}{section}{СПИСОК СОКРАЩЕНИЙ}
    \section*{СПИСОК СОКРАЩЕНИЙ}
    
    \begin{enumerate}
        \item[] ARIS - architecture of integrated information system.
        \item[] ИС - информационная система.
        \item[] МОЛ - материально отвественное лицо.
        \item[] ОА - объект автоматизации.
        \item[] ОП - оперативный документ.
        \item[] ОТ - отчётный документ.
        \item[] СП - справочный документ.
        \item[] СУБД - система управления базами данных.
        \item[] ТМЦ - товарно-материальные ценности.
        \item[] БД - база данных.
    \end{enumerate}
    \newpage
\end{document}
